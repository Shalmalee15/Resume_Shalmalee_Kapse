%%%%%%%%%%%%%%%%%
% This is an sample CV template created using altacv.cls
% (v1.1.4, 27 July 2018) written by LianTze Lim (liantze@gmail.com). Now compiles with pdfLaTeX, XeLaTeX and LuaLaTeX.
% 
%% It may be distributed and/or modified under the
%% conditions of the LaTeX Project Public License, either version 1.3
%% of this license or (at your option) any later version.
%% The latest version of this license is in
%%    http://www.latex-project.org/lppl.txt
%% and version 1.3 or later is part of all distributions of LaTeX
%% version 2003/12/01 or later.
%%%%%%%%%%%%%%%%

%% If you need to pass whatever options to xcolor
\PassOptionsToPackage{dvipsnames}{xcolor}

%% If you are using \orcid or academicons
%% icons, make sure you have the academicons 
%% option here, and compile with XeLaTeX
%% or LuaLaTeX.
% \documentclass[10pt,a4paper,academicons]{altacv}

%% Use the "normalphoto" option if you want a normal photo instead of cropped to a circle
% \documentclass[10pt,a4paper,normalphoto]{altacv}

\documentclass[10pt,a4paper]{altacv}
%% AltaCV uses the fontawesome and academicon fonts
%% and packages. 
%% See texdoc.net/pkg/fontawecome and http://texdoc.net/pkg/academicons for full list of symbols.
%% 
%% Compile with LuaLaTeX for best results. If you
%% want to use XeLaTeX, you may need to install
%% Academicons.ttf in your operating system's font 
%% folder.


% Change the page layout if you need to
\geometry{left=1cm,right=9cm,marginparwidth=6.8cm,marginparsep=1.2cm,top=1.25cm,bottom=1.25cm,footskip=2\baselineskip}

% Change the font if you want to.

% If using pdflatex:
\usepackage[T1]{fontenc}
\usepackage[utf8]{inputenc}
\usepackage[default]{lato}
\newcommand{\latex}{\LaTeX\xspace}
% If using xelatex or lualatex:
% \setmainfont{Lato}

% Change the colours if you want to
\definecolor{Navy}{HTML}{000080}
\definecolor{SlateGrey}{HTML}{2E2E2E}
\definecolor{LightGrey}{HTML}{666666}
\colorlet{heading}{Navy}
\colorlet{accent}{Navy}
\colorlet{emphasis}{SlateGrey}
\colorlet{body}{LightGrey}


% Change the bullets for itemize and rating marker
% for \cvskill if you want to
\renewcommand{\itemmarker}{{\small\textbullet}}
\renewcommand{\ratingmarker}{\faCircle}
%% sample.bib contains your publications
\addbibresource{sample.bib}

\usepackage{hyperref}

\begin{document}

\name{SHALMALEE KAPSE }
\tagline{PhD Candidate in Data Intensive Astrophysics}
\personalinfo{%
  % Not all of these are required!
  % You can add your own with \printinfo{symbol}{detail}
  \email{kapse.shalmalee@gmail.com }
  \phone{+61 406583748} 
%  \homepage{www.homepage.com}
%  \twitter{@twitterhandle}
  \linkedin{\href{https://www.linkedin.com/in/shalmalee-kapse}{linkedin.com/in/shalmalee-kapse}}
  \github{\href{https://www.github.com/Shalmalee15}{github.com/Shalmalee15}}
  %% You MUST add the academicons option to \documentclass, then compile with LuaLaTeX or XeLaTeX, if you want to use \orcid or other academicons commands.
%   \orcid{orcid.org/0000-0000-0000-0000}
}

%% Make the header extend all the way to the right, if you want. 
\begin{fullwidth}
\makecvheader
\end{fullwidth}

%% Depending on your tastes, you may want to make fonts of itemize environments slightly smaller
% \AtBeginEnvironment{itemize}{\small}


%% Provide the file name containing the sidebar contents as an optional parameter to \cvsection.
%% You can always just use \marginpar{...} if you do
%% not need to align the top of the contents to any
%% \cvsection title in the "main" bar.
\cvsection[page1sidebar]{Experience}
\cvevent{Researcher}{Macquarie University}{May 2019 -- Present}{Sydney, Australia}
\begin{itemize}
\item Classified star clusters based on chemical compositions    
\item Developed classification techniques for data from Hubble Space Telescope using \textbf{Python} \textit{(Numpy, Pandas, Matplotlib, Scipy)} 
\item Used \textbf{Scikit-Learn}  to implement Machine Learning algorithms for pattern recognition in astrophysical data 
\item Qualified in implementing calibration pipeline (HSTCAL) in my research 
\end{itemize}
\divider
\cvevent{Casual Academic}{Macquarie University}{October 2019 -- Present}{Sydney, Australia}
\begin{itemize}
\item Supervisor for basic Physics and Astrophysics  
\item Taught 88 undergraduate students in each class
\end{itemize}
\divider
\cvevent{Research Assistant}{Centre for Modeling and Simulation }{March 2017 -- February 2019}{Pune, India}
\begin{itemize}
%\item Predicted a new phase of rock in the Earth's interior using the first principle methods in computational physics 
\item Implemented \textbf{Evolutionary Algorithm} for prediction of new chemical compositions in rocks in Earth's interior
\end{itemize}
\divider
\cvevent{Physics Tutor}{Chegg}{February 2016 -- February 2017}{Pune, India}
\begin{itemize}
\item Delivered lectures on Statistical Physics, Thermodynamics, Classical Physics to college students from the United States 
\end{itemize}
\divider
\cvevent{Internships}{Department of Atomic Energy}{August -- September 2018}{Indore, India}
\begin{itemize}
\item Analysed healthcare data using numerical methods in C
\end{itemize}
\cvevent{}{Inter University Centre for Astronomy and Astrophysics}{May -- December 2015}{Pune, India}
\begin{itemize}
\item Applied Leapfrog and Runge-Kutta methods in C to analyse the effect of interaction between galactic disks and molecular clouds
\end{itemize}
\cvsection{Education}
\cvevent{PhD in Astrophysics}{Macquarie University, Australia}{2019 -- 2022}{}
\vspace{0.05cm}
\cvevent{BSc and MSc in Physics}{Pune University, India}{2009 -- 2015}{}
\end{document}





